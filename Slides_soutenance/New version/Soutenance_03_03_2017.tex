
\documentclass{bredelebeamer}
\usepackage{textcomp}
\usepackage[utf8]{inputenc}
\usepackage[T1]{fontenc}
\usepackage{helvet}
\usepackage{wrapfig}
\usepackage{graphicx}
\graphicspath{ {images/} }
%\usepackage[main=english, french]{babel}

\AtBeginSection[]
{
  \begin{frame}<beamer>
    \frametitle{Part \thesection}
    \tableofcontents[currentsection]
  \end{frame}
}


%%%%%%%%%%%%%%%%%%%%%%%%%%%%%%%%%%%%%%%%%%%%%%%%



\title[Long Project with Audiogaming]{Long Project with Audiogaming}
% Titre du diaporama

\subtitle{Additive Synthesis with Inverse Fourier Transform for Non-Stationary Signals }
% Sous-titre optionnel

\author{ \hspace{0.3cm} C. Cazorla - V. Chrun - B. Fundaro - C. Maliet \hspace{0.3cm} }
% La commande \inst{...} Permet d'afficher l' affiliation de l'intervenant.
% Si il y a plusieurs intervenants: Marcel Dupont\inst{1}, Roger Durand\inst{2}
% Il suffit alors d'ajouter un autre institut sur le modèle ci-dessous.

\institute[Audiogaming Supervisor ]
{
  \normalsize Audiogaming Supervisor : \\
 \normalsize Chunghsin Yeh
  }


\date{March 03, 2017}
% Optionnel. La date, généralement celle du jour de la conférence

\subject{Long Project with Audiogaming}
% C'est utilisé dans les métadonnes du PDF


\logo{
\includegraphics[scale=0.08]{enseeiht.png}
%\includegraphics[scale=0.15]{ag.png}
}



%%%%%%%%%%%%%%%%%%%%%%%%%%%%%%%%%%%%%%%%%%%%%%%%%%%%%%%%%%%%%%%%%%%%%
\begin{document}

\begin{frame}
  \titlepage
\end{frame}


%%%%%%%%%%%%%%%%%


\begin{frame}{Content}
  \tableofcontents
\end{frame}

%-------------------------------------------------------
\section{Introduction}
%-------------------------------------------------------
\subsection{The company}
\begin{frame}{Introduction}{The company}
%-------------------------------------------------------
	\begin{figure}
   	 \centering
  	 \includegraphics[scale=0.12]{ag.png}
	 \end{figure}
  \begin{itemize}
  \item<1-> Localization: Toulouse, Paris
  \item<1-> Activity: Audio plug-in (VSTs and RTAS)
  \item<1-> Main customers: Film and Video Game Industry (Sony, Ubisoft)
  \item<1-> 10 employees
	\begin{figure}
	\includegraphics[scale=0.12]{AudioFire_screen.png}
	\caption{Audiofire: audio plug-in that recreates fire sound}
	\end{figure}
  \end{itemize}
\end{frame}

%-------------------------------------------------------
\subsection{Objective}
\begin{frame}{Introduction}{Objective}
%-------------------------------------------------------


  \begin{itemize}
          \item<1-> We are continuing the Audiogaming long project from 2015 (Emilie Abia, Lili Zheng, Quentin Biache)
\begin{block}{}
\begin{center} {\it Objective} : Synthesizing sounds from their spectrum with a $ FFT^{-1} $ \end{center}
	\begin{figure}
	\includegraphics[scale=0.4]{Analysis_Synthesis.png}
	\caption{General approach for modifying a sound in the spectral domain}
	\end{figure}
\end{block}   
	\item<1-> We have to implement a new method of additive synthesis $\Rightarrow$ computationally very fast
  \end{itemize}
\end{frame}

%-------------------------------------------------------
\subsection{Context of the Project}
\begin{frame}{Introduction}{Context of the Project}
%-------------------------------------------------------

  \begin{itemize}
    \item<1-> 6 weeks only $\Rightarrow$ Focus on the synthesis method only.
   \begin{block}{}
	 Given codes in Python and Matlab from the 2015 project :
   \begin{itemize}
 	\item<1-> Python : Analysis estimator of sinus parameters and sinus generation with those parameters (only stationary)
 	\item<1-> Matlab : Some reasearch on the Non-stationary synthesis with the LUT of lobes
   \end{itemize}
   \end{block}   
	\item<1-> We made our own Object Oriented Programmation tree structure in Python
	\item<1-> We remade all the codes to be coherent with the OOP tree structure
  \end{itemize}
\end{frame}

%-------------------------------------------------------
\subsection{Work Environment and Project Management}
\begin{frame}{Introduction}{Work Environment}
%-------------------------------------------------------

\begin{figure}
	\centering
	\includegraphics[scale=0.4]{all_softwares.png}
	\caption{ {\it PyCharm} as Python IDE , {\it Slack} to communicate,{\it GitHub} to stock the codes and have a versionning, {\it Freedcamp} to plan the project events }
\end{figure}
\end{frame}

\begin{frame}{Introduction}{Project Management : Gantt Chart (expected event)}
%-------------------------------------------------------

\begin{figure}
	\centerline{\includegraphics[scale=0.28]{Gantt.png}}
\end{figure}
\end{frame}

\begin{frame}{Introduction}{Project Management : Gantt Chart now}
%-------------------------------------------------------

\begin{figure}
	\centerline{\includegraphics[scale=0.28]{Gantt.png}}
\end{figure}
\end{frame}

%%%%%%%%%%%%%%%%%
%-------------------------------------------------------
\section{Method Overview}
%-------------------------------------------------------
\subsection{Additive Synthesis (Time Domain)}
\begin{frame}{Additive Synthesis}{Time Domain}
%-------------------------------------------------------
\begin{block}{}
The sound signal is represented as a sum of N sinusoids: \\
\centerline{
$ x(t) = \sum\limits_{n=1}^N a_{n} sin(2 \pi f_{n} t + \phi_{n})$}
\begin{itemize}
\item Very costly to implement
\item Impossible to compute in real-time
\end{itemize}
\end{block}
\begin{figure}
	\centerline
	{\includegraphics[scale=0.25]{additif.png}}
	\caption{\it The additive synthesis}
\end{figure}
\end{frame}

%-------------------------------------------------------
\subsection{Method Overview}
\begin{frame}{Method Overview : Windowing}{Analysis}
%-------------------------------------------------------
\begin{figure}
	\centerline
	{\includegraphics[scale=0.4]{slide1.png}}
	\caption{\it Windowing step}
\end{figure}
\end{frame}

\begin{frame}{Method Overview : Peak detection in Frequency Domain}{Analysis}
%-------------------------------------------------------
Peak detection and extraction of parameters by STPT (particular Short Time Fourier Transform):
\begin{figure}
	\centerline
	{\includegraphics[scale=0.5]{slide2.png}}
	\caption{\it Peak detection}
\end{figure}
\end{frame}

\begin{frame}{Method Overview : Result ( FFT$^{-1}$)}{Synthesis}
%-------------------------------------------------------
Additive synthesis with FFT$^{-1}$ according to the parameters from the analysis:
\begin{figure}
	\centerline
	{\includegraphics[scale=0.5]{synthesisstep.png}}
	\caption{\it Synthesized frame vs Original frame}
\end{figure}
\end{frame}


%%%%%%%%%%%%%%%%%
%-------------------------------------------------------
\section{The additive synthesis in frequency domain}
%-------------------------------------------------------
\subsection{Stationary Case}
\begin{frame}{Stationary Case}{Stationary sinusoidal model}
%-------------------------------------------------------
\begin{block}{}
Mathematical model : \\

\begin{equation}
s(t) = a_0 exp[j(2 \pi f_0 t + \phi_0)]
\end{equation}
\end{block}
\begin{itemize}
\item 3 parameters: $a_0$ (amplitude), $f_0$ (frequency) and $\phi_0$ (phase).
\item Simplest model but useful for certain kinds of signals.
\item Each spectral bin represents a stationary sinusoid.
\end{itemize}

\end{frame}
%-------------------------------------------------------
\begin{frame}{Stationary Case}{Lobe generation}
%-------------------------------------------------------
\begin{block}{}
We generate the sinusoids in frequency domain in order to reduce the computation time : \\

\begin{itemize}
\item Window the signal to maximize the energy in the main lobe
\item We only keep the main lobe for each sine (11 points)
\item We assume that the parameters (amplitude, frequency, phase) are already given by the analysis
\end{itemize}
\end{block}
\begin{figure}
	\centerline
	{\includegraphics[scale=0.75]{lobe.png}}
	\caption{\it Windowed sine lobe}
\end{figure}
\end{frame}
%-------------------------------------------------------
\begin{frame}{Stationary Case}{Frames separation}
%-------------------------------------------------------

The sound signal is  a frame-by-frame signal: \\

\begin{figure}

	{\includegraphics[scale=0.3]{overlap2.png}}
	\caption{\it Sum of small size Hanning windows}
\end{figure}
\begin{figure}
	{\includegraphics[scale=0.25]{overlap1.png}}
	\caption{\it Overlap and add}
\end{figure}
\end{frame}
%-------------------------------------------------------
\begin{frame}{Stationary Case}{Phase coherence}
%-------------------------------------------------------

Phase coherence ???
\end{frame}
%-------------------------------------------------------
\subsection{Casi-Stationary Case}
\begin{frame}{Casi-Stationary Case}{What is changing}
%-------------------------------------------------------

\end{frame}
%-------------------------------------------------------
\subsection{Non-Stationary Case}
\begin{frame}{Non-Stationary Case}{Very different approach}
%-------------------------------------------------------

\begin{block}{}
Mathematical model : \\

\begin{equation}
s(t) = exp[(\lambda_0 + \mu_0 t) + j(\phi_0 + 2\pi f_0 t + \frac{\psi_0}2 t^2 )]
\end{equation}
\end{block}
\begin{itemize}
\item 5 parameters: \\ $(\lambda_0 + \mu_0 t)$ (overall amplitude)\\ $f_0$ (frequency)\\ $\phi_0$ (phase) \\ $ \mu_0$ (amplitude change rate (ACR))\\ $\psi_0$ (frequency change rate (FCR))
\item The analysis part give us all those parameters
\item To manage the influence of the ACR and the FCR on the lobe $\Rightarrow$ Interpolation of Look-up table of already saved lobes with different (ACR,FCR).
\end{itemize}

\end{frame}
%-------------------------------------------------------
\begin{frame}{Non-Stationary Case}{Look up table}
%-------------------------------------------------------


\end{frame}
%-------------------------------------------------------
\begin{frame}{Non-Stationary Case}{Phase Vocoder}
%-------------------------------------------------------


\end{frame}


%%%%%%%%%%%%%%%%%
%-------------------------------------------------------
\section{Result}
%-------------------------------------------------------
\subsection{Stationary Case}
\begin{frame}{Stationary Case}{Sine waves}
%-------------------------------------------------------

\end{frame}
%-------------------------------------------------------
\begin{frame}{Stationary Case}{Triangular waves}
%-------------------------------------------------------

\end{frame}
%-------------------------------------------------------
\subsection{Casi-Stationary Case}
\begin{frame}{The additive synthesis}{Changed Sine waves}
%-------------------------------------------------------

\end{frame}
%-------------------------------------------------------
\subsection{Non-Stationary Case}
\begin{frame}{The additive synthesis}{Chirps}
%-------------------------------------------------------
\end{frame}

%%%%%%%%%%%%%%%%%
%-------------------------------------------------------
\section{Conclusion}
%-------------------------------------------------------
\subsection{Conclusion}
\begin{frame}{Conclusion}{}
%-------------------------------------------------------

\end{frame}
%-------------------------------------------------------
\subsection{References}
\begin{frame}{References}{}
%-------------------------------------------------------

\end{frame}



\end{document}
